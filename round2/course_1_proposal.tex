\documentclass[12pt]{article} 

\usepackage{natbib}
\usepackage{graphics}
\usepackage{amsmath, amssymb}

\usepackage{fancyhdr, graphicx}
\renewcommand{\headrulewidth}{0pt}
\setlength{\headheight}{86.31438pt}
\fancyhead[L]{QuantEcon Workshop Proposal}
\fancyhead[R]{
\includegraphics[width=2cm]{qe-logo-large.png}
}
%\pagestyle{plain}

\usepackage[citecolor=blue, colorlinks=true, linkcolor=blue]{hyperref}

\usepackage{geometry}
\geometry{verbose,tmargin=4cm,bmargin=4cm,lmargin=2cm,rmargin=2cm, headheight=40pt, footskip=30pt}

%extra spacing
\renewcommand{\baselinestretch}{1.0}


% mics short cuts and symbols

\newcommand{\st}{\ensuremath{\ \mathrm{s.t.}\ }}
\newcommand{\setntn}[2]{ \{ #1 \, : \, #2 \} }
\newcommand{\fore}{\therefore \quad}

% skip a line between paragraphs, no indentation
\setlength{\parskip}{1.5ex plus0.5ex minus0.5ex}
\setlength{\parindent}{0pt}


\newcommand{\aA}{\mathscr A}
\newcommand{\bB}{\mathscr B}
\newcommand{\mM}{\mathscr M}
\newcommand{\eE}{\mathscr E}
\newcommand{\fF}{\mathscr F}
\newcommand{\dD}{\mathscr D}
\newcommand{\oO}{\mathcal O}

\newcommand{\EE}{\mathbb E}
\newcommand{\RR}{\mathbb R}
\newcommand{\HH}{\mathbb H}
\newcommand{\NN}{\mathbb N}
\newcommand{\KK}{\mathbb K}


%%%%%%%%%%%%%%%%%% end my preamble %%%%%%%%%%%%%%%%%%%%%%%%%%%%%%%%%%%






\begin{document}


\title{Report}

\date{}

%\maketitle

\begin{center}
    {\bf {\Large High Performance Computing Workshops \\
            for the Central Bank of Chile
    \\
    \vspace{1em}
    {\Large Course 1: Introduction to Scientific Computing with Python and Julia}
    \\
    \vspace{1em}
    Provider: QuantEcon \\
    }}
    \vspace{1em}
    \today
\end{center}


\vspace{.01in}




%section



\thispagestyle{fancy}
Course 1 will provide 
%
\begin{enumerate}
    \item 16 hours of in-person teaching, including lectures and
        tutorials.
    \item Non-graded tutorial and homework exercises.
    \item Accompanying Jupyter notebooks containing both code and theory.
    \item Access to a cloud computing option for all workshop participants.
\end{enumerate}

Instructors:
%
\begin{enumerate}
    \item John Stachurski (Australian National University, Co-founder of QuantEcon)
    \item Pablo Winant (CREST and ESCP Business School, lead developer of \texttt{dolo})
\end{enumerate}

Price: $5000$ dollars.

Dates: 
%
\begin{itemize}
    \item Early May 2022, with exact dates to be determined.
\end{itemize}


Topics:
%
\begin{itemize}
    \item Introduction to Python for scientific computing
    \item Remote and cloud computing with Python
    \item NumPy array operations on the CPU
    \item Numerical optimization tools
    \item Introduction to the Numba just-in-time (JIT) compiler
    \item Writing optimized code for Numba
    \item Application: Markov chains and time series models
    \item Application: Distribution dynamics
    \item Application: Search and optimal stopping
    \item Application: Sovereign default models
    \item Application: Default cascades in financial networks
    \item Parallelization on the CPU
    \item Parallelization on the GPU via CUDA
    \item Automatic differentiation and GPU computing with JAX
    \item Introduction to the Julia language
    \item Types, multiple dispatch and the Julia JIT compiler
\end{itemize}



\end{document}













