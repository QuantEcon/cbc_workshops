\documentclass[12pt]{article} 

\usepackage{natbib}
\usepackage{graphics}
\usepackage{amsmath, amssymb}

\usepackage{fancyhdr, graphicx}
\renewcommand{\headrulewidth}{0pt}
\setlength{\headheight}{86.31438pt}
\fancyhead[L]{QuantEcon Workshop Proposal}
\fancyhead[R]{
\includegraphics[width=2cm]{qe-logo-large.png}
}
%\pagestyle{plain}

\usepackage[citecolor=blue, colorlinks=true, linkcolor=blue]{hyperref}

\usepackage{geometry}
\geometry{verbose,tmargin=4cm,bmargin=4cm,lmargin=2cm,rmargin=2cm, headheight=40pt, footskip=30pt}

%extra spacing
\renewcommand{\baselinestretch}{1.0}


% mics short cuts and symbols

\newcommand{\st}{\ensuremath{\ \mathrm{s.t.}\ }}
\newcommand{\setntn}[2]{ \{ #1 \, : \, #2 \} }
\newcommand{\fore}{\therefore \quad}

% skip a line between paragraphs, no indentation
\setlength{\parskip}{1.5ex plus0.5ex minus0.5ex}
\setlength{\parindent}{0pt}


\newcommand{\aA}{\mathscr A}
\newcommand{\bB}{\mathscr B}
\newcommand{\mM}{\mathscr M}
\newcommand{\eE}{\mathscr E}
\newcommand{\fF}{\mathscr F}
\newcommand{\dD}{\mathscr D}
\newcommand{\oO}{\mathcal O}

\newcommand{\EE}{\mathbb E}
\newcommand{\RR}{\mathbb R}
\newcommand{\HH}{\mathbb H}
\newcommand{\NN}{\mathbb N}
\newcommand{\KK}{\mathbb K}


%%%%%%%%%%%%%%%%%% end my preamble %%%%%%%%%%%%%%%%%%%%%%%%%%%%%%%%%%%






\begin{document}


\title{Report}

\date{}

%\maketitle

\begin{center}
    {\bf {\Large High Performance Computing Workshops \\
            for the Central Bank of Chile
    \\
    \vspace{1em}
    {\Large Course 1: Introduction to Scientific Computing with Python and Julia}
    \\
    \vspace{1em}
    Provider: QuantEcon \\
    }}
    \vspace{1em}
    \today
\end{center}


\vspace{.01in}




%section



\thispagestyle{fancy}
Course 1 will provide 
%
\begin{enumerate}
    \item In-person teaching, including lectures and
        tutorials.
    \item Non-graded tutorial and homework exercises.
    \item Accompanying Jupyter notebooks containing both code and theory.
    \item Access to a cloud computing option for all workshop participants.
\end{enumerate}

Instructors:
%
\begin{enumerate}
    \item John Stachurski (Australian National University, Co-founder of QuantEcon)
    \item Pablo Winant (CREST and ESCP Business School, lead developer of \texttt{dolo})
\end{enumerate}

Dates: 
%
\begin{itemize}
    \item September 20th-23rd 2022.
\end{itemize}

Daily format:
%
\begin{itemize}
    \item 08:30 - 10:30: Lecture
    \item 10:30 - 11:00: Coffee Break
    \item 11:00 - 13:00: Practice Sessions
    \item 13:00 - 14:30: Lunch (at Central Bank offices)
    \item 14:30 - 16:00: Office hours
\end{itemize}

Price: $5000$ USD.

Topics:
%
\begin{enumerate}
    \item Python for scientific computing
    \item NumPy array operations on the CPU
    \item Introduction to the Numba just-in-time (JIT) compiler
    \item Application: Markov chains, time series models and distribution dynamics
    \item Application: Search and optimal stopping
    \item Application: Asset pricing
    \item Application: Dynamic programming
    \item Application: Default cascades in financial networks
    \item Parallelization on the CPU
    \item Parallelization on the GPU via CUDA
    \item Automatic differentiation and GPU computing with JAX
    \item Introduction to deep learning methods in Python
\end{enumerate}



\end{document}













